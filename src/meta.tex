%\documentclass[hyperref={pdfpagelabels=false}]{beamer}
%\documentclass[notes=only,xcolor=svgnames,professionalfonts,lualatex]{beamer}
\documentclass[notes=hide,xcolor=svgnames,professionalfonts,lualatex]{beamer}
\mode<presentation>
{
 \usetheme{Dublin} 
}
\usepackage[british]{babel}
\usepackage{fontspec}
\usepackage{mathtools}
\usepackage{listings}
\usepackage[math-style=french,bold-style=upright, nabla=upright,warnings-off={mathtools-colon}]{unicode-math}%
\usepackage{soul}
\usepackage{extramath}
\usepackage{tikz}%
\usetikzlibrary{chains,matrix,positioning,scopes,shapes,shadows,pgfplots.groupplots,decorations.pathmorphing}
\usetikzlibrary{arrows, decorations.markings,fadings}
\usepackage{pgfplots}
\pgfplotsset{compat=newest}
\usepackage{chemarrow}
%\usepackage[final]{movie15}
\usepackage{setspace}
%\usepackage{media9}
\usepackage{pgfpages}
\usepackage{minted}
\uselanguage{BritishEnglish}
\pgfplotsset{compat=newest}
\usepackage[absolute,overlay]{textpos}

\AtBeginDocument{
\setmathfont{New Euler}%
\fontspec[OpticalSize=15]{New Euler}%
\setmainfont[ItalicFont={OpenSans Light Italic},BoldFont={OpenSans},
            BoldItalicFont={OpenSans Italic},Ligatures=TeX]{OpenSans Light}%
\setsansfont[ItalicFont={OpenSans Light Italic},BoldFont={OpenSans},
            BoldItalicFont={OpenSans Italic},Ligatures=TeX]{OpenSans Light}%
\setmonofont[ItalicFont={OpenSans Italic},BoldFont={OpenSans Bold},
            BoldItalicFont={OpenSans Bold Italic},Ligatures=TeX]{OpenSans}%
\setmathfont[range=\mathup/{latin,Latin,num}]{OpenSans Light}%
\setmathfont[range=\mathup/{greek,Greek}]{OpenSans Light}%
\setmathfont[range=\mathbb/{latin,Latin,num}]{OpenSans }%
\setmathfont[range=\mathbf/{latin,Latin,num}]{OpenSans }
\setmathfont[range=\mathbfup/{latin,Latin,num}]{OpenSans }
\setmathfont[range=\mathbfit/{latin,Latin,num}]{OpenSans Italic}
\setmathfont[range=\mathit/{latin,Latin,num}]{OpenSans Light}
\setmathfont[range=\mathtt/{latin,Latin,num}]{OpenSans}
\setmathfont[range=\mathrm]{OpenSans Light}
\setmathfont[range=\mathsfit/{latin,Latin,num}]{OpenSans Light}
\setmathfont[range=\mathsf/{latin,Latin,num}]{OpenSans Light}
\setmathfont[range={"002C,"0021,"003A,"003B,"220F,"2206,"2202}]{OpenSans Light}
\setmathfont[range={"226A,"226B,"00B0,"03F5,"299E,"03D5,"2218,"2221}]{Asana Math}
\setmathfont[range={\mathcal,\mathbfcal,\mathscr},Scale=1.04]{New Euler}
\setmathfont[range={"03B1,"03B3,"2211,"220F,"2212,"002B}]{New Euler}
\DeclareMathSizes{12}{12}{9}{7}
}

%\setbeameroption{show notes on second screen=right}
\graphicspath{{figures/}}
%\setbeamertemplate{navigation symbols}{}
\setbeamertemplate{sidebar right}{}% or get rid of navigation entries there somehow from the default position... 
%\setbeameroption{show notes on second screen}
\newcommand{\captionfont}{\footnotesize}
\title[Metaprogramming]
{Metaprogramming}
\author[Christian Lalanne \emph{et. al}] % (optional, use only with lots of authors)
{\textcolor{CherryRed}{Christian Lalanne}}
% - Give the names in the same order as the appear in the paper.
% - Use the \inst{?} command only if the authors have different
%   affiliation.

\institute[ICHEC] % (optional, but mostly nee)
{
    %\inst{1}%
     \emph{Irish Centre for High-End Computing}    
}
% - Use the \inst command only if there are several affiliations.
% - Keep it simple, no one is interested in your street address.

\date[October 3, 2014] % (optional, should be abbreviation of conference nam
{Dublin\ October 3, 2014}
% - Either use conference name or its abbreviation.
% - Not really informative to the audience, more for people (including
%   yourself) who are reading the slides online

\subject{Meta\-programming}
% This is only inserted into the PDF information catalogue. Can be left
% out. 

% If you have a file called "university-logo-filename.xxx", where xxx
% is a graphic format that can be processed by latex or pdflatex,
% resp., then you can add a logo as follows:

%  \pgfdeclareimage[height=0.88cm]{university-logo}{ucdlogo}
%   \logo{\pgfuseimage{university-logo}}
\pgfdeclareimage[height=0.9cm]{group-logo}{figures/ichec}
\logoM{\pgfuseimage{group-logo}}


%%%%%%%%%%%%%%%%%%%%
\makeatletter
\def\pgf@plot@curveto@handler@finish{%
  \ifpgf@plot@started%

\pgfpathcurvebetweentimecontinue{0}{0.995}{\pgf@plot@curveto@first}{
\pgf@plot@curveto@first@support}{\pgf@plot@curveto@second}{
\pgf@plot@curveto@second}%
  \fi%
}
\makeatother

\makeatletter
\def\pgf@plot@curveto@handler@finish{%
  \ifpgf@plot@started%

\pgfpathcurvebetweentimecontinue{0}{0.995}{\pgf@plot@curveto@first}{
\pgf@plot@curveto@first@support}{\pgf@plot@curveto@second}{
\pgf@plot@curveto@second}%
  \fi%
}
\makeatother


\tikzfading[name=arrowfading, top color=transparent!0, bottom color=transparent!95]
\tikzset{arrowfill/.style={#1,general shadow={fill=black, shadow yshift=-0.8ex, path fading=arrowfading}}}
\tikzset{arrowstyle/.style n args={3}{draw=#2,arrowfill={#3}, double arrow,minimum height=#1,double arrow head extend=.3cm}}

\NewDocumentCommand{\tikzfancyarrow}{O{2cm} O{red!50} O{top color=blue!20, bottom color=blue!50} m}{
\tikz[baseline=-0.5ex]\node [arrowstyle={#1}{#2}{#3}] {#4};
}


\newif\ifcuboidshade
\newif\ifcuboidemphedge

\tikzset{
  cuboid/.is family,
  cuboid,
  shiftx/.initial=0,
  shifty/.initial=0,
  dimx/.initial=3,
  dimy/.initial=3,
  dimz/.initial=3,
  scale/.initial=1,
  densityx/.initial=1,
  densityy/.initial=1,
  densityz/.initial=1,
  rotation/.initial=0,
  anglex/.initial=0,
  angley/.initial=90,
  anglez/.initial=225,
  scalex/.initial=1,
  scaley/.initial=1,
  scalez/.initial=0.5,
  front/.style={draw=black,fill=white},
  top/.style={draw=black,fill=white},
  right/.style={draw=blue!25!black,fill=blue!25!black},
  shade/.is if=cuboidshade,
  shadecolordark/.initial=blue!75!black,
  shadecolorlight/.initial=blue!50!black,
  shadeopacity/.initial=0.15,
  shadesamples/.initial=64,
  emphedge/.is if=cuboidemphedge,
  emphstyle/.style={thick},
  labelcolour/.initial=white,
  label/.initial="",
  lx/.initial=0,
  ly/.initial=0,
}

\newcommand{\tikzcuboidkey}[1]{\pgfkeysvalueof{/tikz/cuboid/#1}}

% Commands
\newcommand{\tikzcuboid}[1]{
    \tikzset{cuboid,#1} % Process Keys passed to command
  \pgfmathsetlengthmacro{\vectorxx}{\tikzcuboidkey{scalex}*cos(\tikzcuboidkey{anglex})*28.452756}
  \pgfmathsetlengthmacro{\vectorxy}{\tikzcuboidkey{scalex}*sin(\tikzcuboidkey{anglex})*28.452756}
  \pgfmathsetlengthmacro{\vectoryx}{\tikzcuboidkey{scaley}*cos(\tikzcuboidkey{angley})*28.452756}
  \pgfmathsetlengthmacro{\vectoryy}{\tikzcuboidkey{scaley}*sin(\tikzcuboidkey{angley})*28.452756}
  \pgfmathsetlengthmacro{\vectorzx}{\tikzcuboidkey{scalez}*cos(\tikzcuboidkey{anglez})*28.452756}
  \pgfmathsetlengthmacro{\vectorzy}{\tikzcuboidkey{scalez}*sin(\tikzcuboidkey{anglez})*28.452756}
  \begin{scope}[xshift=\tikzcuboidkey{shiftx}, yshift=\tikzcuboidkey{shifty}, scale=\tikzcuboidkey{scale}, rotate=\tikzcuboidkey{rotation}, x={(\vectorxx,\vectorxy)}, y={(\vectoryx,\vectoryy)}, z={(\vectorzx,\vectorzy)}]
    \pgfmathsetmacro{\steppingx}{1/\tikzcuboidkey{densityx}}
  \pgfmathsetmacro{\steppingy}{1/\tikzcuboidkey{densityy}}
  \pgfmathsetmacro{\steppingz}{1/\tikzcuboidkey{densityz}}
  \newcommand{\dimx}{\tikzcuboidkey{dimx}}
  \newcommand{\dimy}{\tikzcuboidkey{dimy}}
  \newcommand{\dimz}{\tikzcuboidkey{dimz}}
  \pgfmathsetmacro{\secondx}{2*\steppingx}
  \pgfmathsetmacro{\secondy}{2*\steppingy}
  \pgfmathsetmacro{\secondz}{2*\steppingz}

  \foreach \x in {\steppingx,\secondx,...,\dimx}
  {\foreach \y in {\steppingy,\secondy,...,\dimy}
    {\pgfmathsetmacro{\lowx}{(\x-\steppingx)}
     \pgfmathsetmacro{\lowy}{(\y-\steppingy)}
     \filldraw[cuboid/front] (\lowx,\lowy,\dimz) -- (\lowx,\y,\dimz) -- (\x,\y,\dimz) -- (\x,\lowy,\dimz) -- cycle;
    }
   }
   \draw[rotate=90] (0.4*\dimx+\tikzcuboidkey{lx},0.4*\dimy+\tikzcuboidkey{ly}) node[\tikzcuboidkey{labelcolour}]{\tikzcuboidkey{label}};
   \foreach \x in {\steppingx,\secondx,...,\dimx}
   {\foreach \z in {\steppingz,\secondz,...,\dimz}
    {\pgfmathsetmacro{\lowx}{(\x-\steppingx)}
     \pgfmathsetmacro{\lowz}{(\z-\steppingz)}
     \filldraw[cuboid/top] (\lowx,\dimy,\lowz) -- (\lowx,\dimy,\z) -- (\x,\dimy,\z) -- (\x,\dimy,\lowz) -- cycle;
    }
   }

    \foreach \y in {\steppingy,\secondy,...,\dimy}
    {\foreach \z in {\steppingz,\secondz,...,\dimz}
     {\pgfmathsetmacro{\lowy}{(\y-\steppingy)}
      \pgfmathsetmacro{\lowz}{(\z-\steppingz)}
      \filldraw[cuboid/right] (\dimx,\lowy,\lowz) -- (\dimx,\lowy,\z) -- (\dimx,\y,\z) -- (\dimx,\y,\lowz) -- cycle;
     }
    }

  \ifcuboidemphedge
    \draw[cuboid/emphstyle] (0,\dimy,0) -- (\dimx,\dimy,0) -- (\dimx,\dimy,\dimz) -- (0,\dimy,\dimz) -- cycle;%
    \draw[cuboid/emphstyle] (0,\dimy,\dimz) -- (0,0,\dimz) -- (\dimx,0,\dimz) -- (\dimx,\dimy,\dimz);%
    \draw[cuboid/emphstyle] (\dimx,\dimy,0) -- (\dimx,0,0) -- (\dimx,0,\dimz);%
    \fi

    \ifcuboidshade
    \pgfmathsetmacro{\cstepx}{\dimx/\tikzcuboidkey{shadesamples}}
    \pgfmathsetmacro{\cstepy}{\dimy/\tikzcuboidkey{shadesamples}}
    \pgfmathsetmacro{\cstepz}{\dimz/\tikzcuboidkey{shadesamples}}
    \foreach \s in {1,...,\tikzcuboidkey{shadesamples}}
    {   \pgfmathsetmacro{\lows}{\s-1}
        \pgfmathsetmacro{\cpercent}{(\lows)/(\tikzcuboidkey{shadesamples}-1)*100}
        \fill[opacity=\tikzcuboidkey{shadeopacity},color=\tikzcuboidkey{shadecolorlight}!\cpercent!\tikzcuboidkey{shadecolordark}] (0,\s*\cstepy,\dimz) -- (\s*\cstepx,\s*\cstepy,\dimz) -- (\s*\cstepx,0,\dimz) -- (\lows*\cstepx,0,\dimz) -- (\lows*\cstepx,\lows*\cstepy,\dimz) -- (0,\lows*\cstepy,\dimz) -- cycle;
        \fill[opacity=\tikzcuboidkey{shadeopacity},color=\tikzcuboidkey{shadecolorlight}!\cpercent!\tikzcuboidkey{shadecolordark}] (0,\dimy,\s*\cstepz) -- (\s*\cstepx,\dimy,\s*\cstepz) -- (\s*\cstepx,\dimy,0) -- (\lows*\cstepx,\dimy,0) -- (\lows*\cstepx,\dimy,\lows*\cstepz) -- (0,\dimy,\lows*\cstepz) -- cycle;
        \fill[opacity=\tikzcuboidkey{shadeopacity},color=\tikzcuboidkey{shadecolorlight}!\cpercent!\tikzcuboidkey{shadecolordark}] (\dimx,0,\s*\cstepz) -- (\dimx,\s*\cstepy,\s*\cstepz) -- (\dimx,\s*\cstepy,0) -- (\dimx,\lows*\cstepy,0) -- (\dimx,\lows*\cstepy,\lows*\cstepz) -- (\dimx,0,\lows*\cstepz) -- cycle;
    }
    \fi 

  \end{scope}
}

\makeatother



% Delete this, if you do not want the table of contents to pop up at
% the beginning of each subsection:
 \AtBeginSubsection[]
 {
   \begin{frame}<beamer>
    \frametitle{Outline}
  \begin{tikzpicture}

  \end{tikzpicture}
    \tableofcontents[currentsection,currentsubsection]
    \addtocounter{framenumber}{-1}
   \note<+->{.}
   \end{frame}
 }


% If you wish to uncover everything in a step-wise fashion, uncomment
% the following command: 

% \beamerdefaultoverlayspecification{<+->}
 
% \tikzstyle{blockt} = [rectangle, text width=15em, text centered, minimum 
% height=2em,inner sep=0pt]
% \tikzstyle{blockline} = [rectangle, text width=\paperwidth, text centered, 
% minimum height=2em,inner sep=0pt]
% \tikzstyle{na} = [baseline=-.5ex]
 
\definecolor{mintedbg}{RGB}{240,248,255}
\definecolor{mintedobg}{RGB}{125,203,51}
\definecolor{ichecGreen}{RGB}{125,203,51}
\definecolor{ichecBlue}{RGB}{11,142,216}
\newminted{sh}{bgcolor=mintedbg,
              linenos=true,
              frame=lines,
              showspaces=false,
              fontsize=\tiny
}

\newminted{fortran}{bgcolor=mintedbg,
              linenos=false,
              frame=lines,
              showspaces=false,
              fontsize=\tiny,
              numbersep=0pt,
              xleftmargin=6pt,
}

\begin{document}
% For every picture that defines or uses external nodes, you'll have to
% apply the 'remember picture' style. To avoid some typing, we'll apply
% the style to all pictures.
\tikzstyle{every picture}+=[remember picture,overlay]

% By default all math in TikZ nodes are set in inline mode. Change this to
% displaystyle so that we don't get small fractions.
\everymath{\displaystyle}
%%%% this is practically the first slide... 
\begin{frame}[plain]
%\vskip-16.4mm
 \begin{beamercolorbox}[wd=1.0\paperwidth,colsep=2.5pt]{upper separation line head}
 \end{beamercolorbox}    
\begin{beamercolorbox}[wd=1.0\paperwidth,colsep=2.5pt]{}
\vbox to 10.70mm {%
 \vfill {\hskip0mm } \vfil%
}%
%\hskip-1.75mm\llap{\includegraphics[height=0.9cm]{sfi}\includegraphics[
% height=0.9cm]{ucdlogo}}
\hskip-1.75mm\llap{\includegraphics[height=0.9cm]{ichec}}
\end{beamercolorbox}

\begin{center}
\begin{beamercolorbox}[wd=1.0\paperwidth,center,ht=13.0mm]{title}
  \begin{minipage}{0.65\paperwidth}
   \begin{center}
    \usebeamerfont{title}\inserttitle
    \end{center}
  \end{minipage}
\end{beamercolorbox}

\begin{beamercolorbox}[wd=\paperwidth,center,ht=11mm]{author}
   \usebeamerfont{author}\insertauthor
\end{beamercolorbox}

\begin{beamercolorbox}[wd=\paperwidth,center,ht=7mm]{institute}
    \usebeamerfont{institute}\insertinstitute
\end{beamercolorbox}

\begin{beamercolorbox}[wd=\paperwidth,center,ht=9mm]{author}
\end{beamercolorbox}

\begin{beamercolorbox}[wd=\paperwidth,center,ht=13mm]{date}
  \usebeamerfont{institute}\insertdate
\end{beamercolorbox}

\end{center}
\addtocounter{framenumber}{-1}
\note<+->{.}
\end{frame}

\begin{frame}{Outline}
  \begin{tikzpicture}[]
  \end{tikzpicture}
  \tableofcontents
  % You might wish to add the option [pausesections]
\addtocounter{framenumber}{-1}
\note<+->{.}
\end{frame}
% Structuring a talk is a difficult task and the following structure
% may not be suitable. Here are some rules that apply for this
% solution: 

% - Exactly two or three sections (other than the summary).
% - At *most* three subsections per section.
% - Talk about 30s to 2min per frame. So there should be between about
%   15 and 30 frames, all told.

% - A conference audience is likely to know very little of what you
%   are going to talk about. So *simplify*!
% - In a 20min talk, getting the main ideas across is hard
%   enough. Leave out details, even if it means being less precise than
%   you think necessary.
% - If you omit details that are vital to the proof/implementation,
%   just say so once. Everybody will be happy with that.
\section{Definitions}
\begin{frame}<beamer>                                                                                                                    
    \frametitle{Outline}
    \tableofcontents[currentsection]
\end{frame}
\begin{frame}{Definitions}
\begin{enumerate}
\item Writing programs that generate or manipulate programs as their data.
\item A program that manipulates code.
\item Code-generating programs are sometimes called meta-programs; writing such programs
    is called meta-programming.
\end{enumerate}
\end{frame}

\begin{frame}{Textual macro languages}
\begin{enumerate}
\item Textual Macros are macros that directly affect the text of the programming 
    language without knowing about or dealing with the meaning of the language \emph{e.g.} C 
    preprocessor and the M4 macro processor. 
\end{enumerate}
\end{frame}

\begin{frame}[fragile]{C preprocessor}
\inputminted[mathescape,
           linenos,
           numbersep=5pt,
           frame=lines,
           bgcolor=White,
           fontsize=\scriptsize,
           linenos,
           framesep=2mm]{c++}
           {/Users/lalanne/MyCode/GitHubProjects/MP_Talk/macro_swap.cpp} 
\end{frame}



\section{Macros}
\begin{frame}<beamer>                                                                                                                    
    \frametitle{Outline}
    \tableofcontents[currentsection]
\end{frame}
\begin{frame}{Macros}
\begin{itemize}
\item Writing programs that generate or manipulate programs as their data.
\item A program that manipulates code.
\item Code-generating programs are sometimes called meta-programs; writing such programs
    is called meta-programming.
\end{itemize}
\end{frame}

\begin{frame}{Textual macro languages}
\begin{itemize}
\item Textual Macros are macros that directly affect the text of the programming 
    language without knowing about or dealing with the meaning of the language \emph{e.g.} C 
    preprocessor and the M4 macro processor \cite{macrosibm}. 
\end{itemize}
\end{frame}

\begin{frame}[fragile]{C preprocessor SWAP}
\inputminted[mathescape,
           linenos,
           numbersep=5pt,
           frame=lines,
           bgcolor=White,
           fontsize=\scriptsize,
           linenos,
           framesep=2mm]{c++}
           {/Users/lalanne/MyCode/GitHubProjects/MP_Talk/macro_swap.cpp} 
\end{frame}

\begin{frame}{Advantages}
\begin{itemize}
\item function call too much overhead for this simple operation (myth).
\item you would have to code different functions for different types.
\end{itemize}
\end{frame}

\begin{frame}{Disadvantages}
\begin{itemize}
\item you do not get any type of type safety.
\item if you pass tmp\_c as an argument the macro will fail.
\end{itemize}
\end{frame}

\begin{frame}[fragile]{C preprocessor MIN}
\inputminted[mathescape,
           linenos,
           numbersep=5pt,
           frame=lines,
           bgcolor=White,
           fontsize=\scriptsize,
           linenos,
           framesep=2mm]{c++}
           {/Users/lalanne/MyCode/GitHubProjects/MP_Talk/macro_min.cpp} 
\end{frame}

\begin{frame}{The MIN case}
\begin{itemize}
\item necessary parenthesis (operator precedence).
\item When one of the arguments is a function is going to be call twice. (performance, duplicated calculation).
\end{itemize}
\end{frame}







\section{C++ Template Metaprogramming}
\begin{frame}<beamer>                                                                                                                    
    \frametitle{Outline}
    \tableofcontents[currentsection]
\end{frame}
\begin{frame}{C++ Template Meta-programming}
    \begin{itemize}
        \item Templates are used by the compiler to create temporary source code
            \cite{wikiTM}.

        \item This code is merge by the compiler with the rest of the source code
            and then compiled\cite{wikiTM}.

        \item Compile time execution\cite{wikiTM}.

        \item C++, D, Lisp(kind of)\cite{wikiTM}.
    \end{itemize}
\end{frame}

\begin{frame}{Numeric Computations, Compile Time}
    \inputminted[mathescape,                                                       
                    linenos,                                                           
                    numbersep=5pt,                                                     
                    frame=lines,                                                       
                    bgcolor=White,                                                     
                    fontsize=\scriptsize,                                              
                    linenos,                                                           
                    framesep=1.5mm]{c++}                                                 
                    {/Users/clalanne/GitHubProjects/MetaTalk/src/code/basic.cpp}
    
\end{frame}

\begin{frame}{Numeric Computations, Run Time}
    \inputminted[mathescape,                                                       
                    linenos,                                                           
                    numbersep=5pt,                                                     
                    frame=lines,                                                       
                    bgcolor=White,                                                     
                    fontsize=\scriptsize,                                              
                    linenos,                                                           
                    framesep=1.5mm]{c++}                                                 
                    {/Users/clalanne/GitHubProjects/MetaTalk/src/code/basic_runtime.cpp}
    
\end{frame}



\section{Erasing virtual methods}
\subsection{Virtual Tables}
\begin{frame}{Dynamic Dispatching}
    \begin{itemize}
        \item When designing a system you usually know which kind of interface
            do you want but not the concrete implementation
        \item The goal is to reduce the effort in changing implementations
        \item \textbf{Dynamic Dispatching:} Which mplementation of a 
            method(class function) to call at run-time
    \end{itemize}
\end{frame}

\begin{frame}{Virtual Functions}
    \begin{center}
        \begin{tikzpicture}[]
            \node[] at (0mm,0mm){
                \includegraphics[height=70mm]{/Users/lalanne/MyCode/GitHubProjects/MetaTalk/figures/ddp1.png}\hspace{5mm}
            };
        \end{tikzpicture}
    \end{center}
\end{frame}

\begin{frame}{Virtual Functions}
    \begin{center}
        \begin{tikzpicture}[]
            \node[] at (0mm,0mm){
                \includegraphics[height=70mm]{/Users/lalanne/MyCode/GitHubProjects/MetaTalk/figures/classdiag.png}\hspace{5mm}
            };
        \end{tikzpicture}
    \end{center}
\end{frame}

\begin{frame}{Virtual Functions}
    \begin{center}
        \begin{tikzpicture}[]
            \node[] at (0mm,0mm){
                \includegraphics[height=70mm]{/Users/lalanne/MyCode/GitHubProjects/MetaTalk/figures/vtbl1.png}\hspace{5mm}
            };
        \end{tikzpicture}
    \end{center}
\end{frame}

\begin{frame}{Virtual Functions}
    \begin{center}
        \begin{tikzpicture}[]
            \node[] at (0mm,0mm){
                \includegraphics[height=70mm]{/Users/lalanne/MyCode/GitHubProjects/MetaTalk/figures/vtbl1_1.png}\hspace{5mm}
            };
        \end{tikzpicture}
    \end{center}
\end{frame}

\begin{frame}{Virtual Functions}
    \begin{center}
        \begin{tikzpicture}[]
            \node[] at (0mm,0mm){
                \includegraphics[height=70mm]{/Users/lalanne/MyCode/GitHubProjects/MetaTalk/figures/vtbl2.png}\hspace{5mm}
            };
        \end{tikzpicture}
    \end{center}
\end{frame}

\begin{frame}{Virtual Functions}
    \begin{center}
        \begin{tikzpicture}[]
            \node[] at (0mm,0mm){
                \includegraphics[height=70mm]{/Users/lalanne/MyCode/GitHubProjects/MetaTalk/figures/vtbl3.png}\hspace{5mm}
            };
        \end{tikzpicture}
    \end{center}
\end{frame}

\begin{frame}{Virtual Functions}
    \begin{center}
        \begin{tikzpicture}[]
            \node[] at (0mm,0mm){
                \includegraphics[height=70mm]{/Users/lalanne/MyCode/GitHubProjects/MetaTalk/figures/vtbl3_1.png}\hspace{5mm}
            };
        \end{tikzpicture}
    \end{center}
\end{frame}

\begin{frame}{Virtual Functions}
    \begin{center}
        \begin{tikzpicture}[]
            \node[] at (0mm,0mm){
                \includegraphics[height=70mm]{/Users/lalanne/MyCode/GitHubProjects/MetaTalk/figures/vtbl4.png}\hspace{5mm}
            };
        \end{tikzpicture}
    \end{center}
\end{frame}


\begin{frame}{Virtual Functions}
    \begin{center}
        \begin{tikzpicture}[]
            \node[] at (0mm,0mm){
                \includegraphics[height=70mm]{/Users/lalanne/MyCode/GitHubProjects/MetaTalk/figures/vtblpluscall.png}\hspace{5mm}
            };
        \end{tikzpicture}
    \end{center}
\end{frame}

\begin{frame}{Virtual Functions}
    \begin{center}
        \begin{tikzpicture}[]
            \node[] at (0mm,0mm){
                \includegraphics[height=70mm]{/Users/lalanne/MyCode/GitHubProjects/MetaTalk/figures/vtblpluscall1.png}\hspace{5mm}
            };
        \end{tikzpicture}
    \end{center}
\end{frame}

\begin{frame}{Virtual Functions}
    \begin{center}
        \begin{tikzpicture}[]
            \node[] at (0mm,0mm){
                \includegraphics[height=70mm]{/Users/lalanne/MyCode/GitHubProjects/MetaTalk/figures/vtblpluscall2.png}\hspace{5mm}
            };
        \end{tikzpicture}
    \end{center}
\end{frame}

\begin{frame}{Virtual Functions}
    \begin{center}
        \begin{tikzpicture}[]
            \node[] at (0mm,0mm){
                \includegraphics[height=70mm]{/Users/lalanne/MyCode/GitHubProjects/MetaTalk/figures/vtblpluscall3.png}\hspace{5mm}
            };
        \end{tikzpicture}
    \end{center}
\end{frame}

\begin{frame}{Virtual Functions}
    \begin{center}
        \begin{tikzpicture}[]
            \node[] at (0mm,0mm){
                \includegraphics[height=70mm]{/Users/lalanne/MyCode/GitHubProjects/MetaTalk/figures/vtblpluscall4.png}\hspace{5mm}
            };
        \end{tikzpicture}
    \end{center}
\end{frame}

\begin{frame}{Virtual Functions}
    \begin{center}
        \begin{tikzpicture}[]
            \node[] at (0mm,0mm){
                \includegraphics[height=70mm]{/Users/lalanne/MyCode/GitHubProjects/MetaTalk/figures/vtblpluscall5.png}\hspace{5mm}
            };
        \end{tikzpicture}
    \end{center}
\end{frame}

\begin{frame}{Virtual Functions}
    \begin{itemize}
        \item First load \emph{vptr} to register (r1)
        \item Second load \emph{vptr + offset} to register (r2)
        \item Call function (r2)
    \end{itemize}

    \begin{itemize}
        \item Small size overhead(think small classes)
        \item Call overhead
    \end{itemize}
\end{frame}


\subsection{crtp}
\begin{frame}{Erasing virtual methods} 
\begin{itemize}
\item .
\end{itemize}
\end{frame}


\section{Motivation}
\begin{frame}<beamer>                                                                                                                    
    \frametitle{Outline}
    \tableofcontents[currentsection]
\end{frame}
\input{motivation}

\section{References}
\input{references}
\section*{Questions}
\begin{frame}{Q\&A}
 \begin{center}
   \begin{tikzpicture}[]
     \node[] at (0mm,0mm){
       \includegraphics[height=30mm]{questionR.jpg}\hspace{5mm}
       \includegraphics[height=30mm]{questionB.jpg}\hspace{5mm}
       \includegraphics[height=30mm]{questionW.png}
     };
   \end{tikzpicture}
 \end{center}
\addtocounter{framenumber}{-1}
\note<+->{.}
\end{frame}

\section{Extra Material}
\input{supp}

\end{document}
